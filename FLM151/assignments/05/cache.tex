\documentclass{article}

\usepackage{times}

\title{Structure of \emph{Cach\'{e}}}

\author{Marcus Meng (FLM151)}

\begin{document}

\maketitle

\emph{Cach\'{e}} is a 2005 movie about a man dealing with the sudden reappearance of a sibling that he wronged in his life.
The film focuses primarily on how the reappearance of his once-to-be sibling affects his relationships with his current family.

Back at the start of the semester, there was a discussion of what makes up a story.
The premise basically ran along the lines of, "A man hears a sound, looks out a window, and sees a woman surrounded by some guys".
It was established that, perhaps, a story would need to consist of more than that basic description;
however, \emph{Cach\'{e}} feels very much like little more than that barebones outline.

It has a very loose structure, at best.
There's an exposition spread throughout the story, and arguably, the scene where Majid kills himself is the climax.
However, while there's some tension build-up as Georges attempts to locate the person sending the mail to him, the film does nothing to resolve that particular thread.
It is easy to see that the director wanted to put focus on the feelings of the various characters as their ties fell apart via trust issues, however, the fact that the threats and stuff played such a large part in the story without being resolved really distracted the viewer from that.

In \emph{Raiders of the Lost Ark}, there's a similarly single-minded attention to the main plot goal of recovering the Ark of the Covenent, but it's hard to say anything particularly notable about it as they also treated the entire story as an appropriate means to that end.
Any digressions were more to add comedic value or a bit of additional action to the story, and things worked out pretty much as expected -- with a bonus of the story actually being resolved.

In terms of managing the story on screen, one would, initially, expect \emph{Cach\'{e}} and \emph{The Verdict} to have similar approaches.
Both being low-concept, heavily character-focused stories, it would have seemed prudent for both directors to adopt converging techniques.
However, Haneke opts instead for a different artistic style -- much of his movie is shot from a farther, unchanging, and unmoving view, reminiscent of a surveillance camera.
\emph{The Verdict} goes for more traditional cuts and camera shots to push the emotions and tensions in the various scenes across, and really, it seems to work better in the end;
\emph{Cach\'{e}}'s artistic style makes the audience feel rather detached from the characters themselves.

\begin{thebibliography}{9}

\bibitem{web:3act}
	\emph{Three-act Structure}, http://www.cod.edu/people/faculty/pruter/film/threeact.htm

\bibitem{web:imdb}
	\emph{Cach\'{e}}, http://www.imdb.com/title/tt0387898/

\bibitem{web:raiders-structure}
	\emph{Structure: Raiders of the Lost Ark}, http://thestorydepartment.com/structure-raiders-of-the-lost-ark/

\end{thebibliography}

\end{document}

