\documentclass{article}

\usepackage{times}

\title{Stories in \emph{Chungking Express}}

\author{Marcus Meng (FLM151)}

\begin{document}

\maketitle

\emph{Chungking Express} is a 1994 film that exists as a set of two stories.
Each part could roughly be described as a relationship story, the first being an almost noir, gritty type of story;
the second being a lighter, somewhat comedic romantic slice-of-life.
In both cases, the main character is some sort of a policeman, though this particular fact is more background and less of any important factor in the stories themselves.

Roughly speaking, the story is paralleled in both halves of the movie -- the director had, in fact, specified that he wanted to tell the same story in two different styles.
As such, not only does the plot share many similarities, the general theme of both parts are quite similar, too:
the stories are all told over time, often showing the characters' day to day lives as well as the specific events that motivate them to act in the story.

However, the details of each story and the style they're presented in, as mentioned, are quite different.
In the first part, the entire story is shot at night, and there's a sense of claustrophobia, crowdedness, and confusion that is heavily influenced by how each scene is shot.
The camera is often angled upwards, very near to people -- there are few, if any, long shots -- which makes the individuals look like they're taking up significantly more space than they actually are.
The shots are almost invasive, like that, and help to give the heavily populated sense that many Chinese cities (the movie was shot in Hong Kong) have.
Action scenes are easily recognizeable -- the camera jumps around jerkily, the scenes are blurred, and the rapidly changing positions of characters on the screen force the viewer to constantly reassess where everything is, giving the events a cramped and rushed feeling that serves to accentuate the confusion that one would experience in the chases through the crowds.

The second part, in contrast to all this, is all shot in broad daylight, with clear medium shots and less people (extras, mostly) on screen at a time.
However, this time around, the camera is placed firmly on the girl, Faye, who has noticed the male lead (an unnamed police officer), and wishes to catch his attention (in her own weird way).
For a large portion of this section, the officer only shows up in between sections; and a lot of it focuses on how Faye passes her time when she sneaks into the officer's house and redecorates his domicile.
The scenes start out relatively drab -- this part, in fact, follows the first part in that the exact same restaraunt shows up -- but as Faye's ninja redecorating starts to take an effect on the officer and his mood brightens, the scenes get brighter, as well.
This progresses more or less as expected until the climax of this story, where the officer is stood up by Faye when he finally realizes her intentions and asks her to meet him at a bar.
Whereupon it goes back to being dark, dimly lit, and rainy.

So, both stories cross, but don't interact, in both literal and figurative senses.
The first story ends in a restaraunt that serves as the central area in the second story, and even the characters cross over to some degree -- Faye is actually introduced at the end of the first part of the story.
The plots and themes cross over as well, both parts dealing with the ability of a man to deal with breakups and the various emotions and thoughts that come with such an experience.
However, each part is distinct -- the only character that actually sees both stories unfold is the restaraunt owner, and he plays such a minor part as to make no odds with regards to the plot's progression.
In a way, the plots don't interact in the very obvious sense -- if the movie were split into two seperate, shorter films, they would still be more or less as coherant as they are now (though, they'd lose the interesting contrast between the two halves).

\begin{thebibliography}{9}

\bibitem{web:imdb}
	\emph{Chungking Express}, http://www.imdb.com/title/tt0109424/

\bibitem{web:erasing-clouds}
	\emph{Cinematic Pleasures: Chungking Express}, http://www.erasingclouds.com/0519chungking.html

\bibitem{web:ebert}
	\emph{Ebert: Chungking Express}, http://rogerebert.suntimes.com/apps/pbcs.dll/article?AID=/19960315/REVIEWS/603150301/1023

\end{thebibliography}

\end{document}

