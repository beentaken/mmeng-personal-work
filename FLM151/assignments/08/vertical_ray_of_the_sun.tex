\documentclass{article}

\usepackage{times}

\title{Poetic Presentation in \emph{The Vertical Ray of the Sun}}

\author{Marcus Meng (FLM151)}

\begin{document}

\maketitle

\emph{The Vertical Ray of the Sun} is a 2000 slice-of-life film that follows the unfolding of a particular set of relationships in a family in Vietnam.
This film has been described as being rather poetic -- and while the literal definition of the word obviously cannot apply to a film, the comparison to poetry is not too far off.

Poetry is defined as the usage of language for aesthetic and evocative qualities in addition to the literal meaning of words, and this film manages to accomplish a similar effect with the images presented.
A common trait in poetry is a certain repetitive structure, that serves to create a consistent sound and rythmn.
\emph{The Vertical Ray of the Sun} manages to do so by substituting in its scene structures, and creating repeating microcosms of the various small actions in the story.
The story covers several years, and each time it jumps and returns to the characters, it comes back to a familiar scene -- Lien and Hai waking up, the sisters getting together and socializing.

It uses this to drive forward progression while keeping the viewer stable.
One can see the story of the various characters real or percieved infidelity without getting mixed up as to when something happened, or that the characters have changed --
a particularly notable feat, since it allows the story to focus a lot on the character's existing relationships and their own thoughts and personalities without confusing anyone with sudden changes of heart or dramatic character development.
This does not mean there's no tension involved: the last few scenes show exactly how characters can bounce of each other, and the scenes reveal much about the character, not directly change how the various characters are.

\begin{thebibliography}{9}

\bibitem{web:imdb}
	\emph{The Vertical Ray of the Sun}, http://www.imdb.com/title/tt0224578/

\bibitem{wiki:poetry}
	\emph{Poetry}, http://en.wikipedia.org/wiki/Poetry

\end{thebibliography}

\end{document}

