\paragraph{The influence of culture and conquest on game development.}
Gaming is a phenomenom not unique to any one era of human history.
There has never been a time where people have not found a time or means to play games, and through this continual usage,
games have accrued traits from the various cultures they pass through as they are handed down through the generations and modified.
This constant accretion of information in games can be a useful tool for anyone studying the history of the cultures involved,
as there are often tidbits of information stored in the mechanics themselves, that are less likely to change than the trappings around them.
\emph{Games arise from common memetics in cultures, and thus, like any creative process, take on aspects of the cultures that create them;
conquests and expanded territories spread games farther than the locales that started them.}

\paragraph{Games, like any other form of art, are heavily influenced by the cultures that created them.}
The oldest formalized games --- that is, ones that tended to get recorded down, back when literacy was expensive and relatively rare ---
tended to be ones directly representing the most obvious aspects of that culture -- religion and nobility, for instance.
These ones are likely to have intentional designs (religious iconography, for instance), but there are others where the ties to the culture are purely incidental (such as tabula's astronomical associations).
Others would be passed on by word of mouth, and changed as they moved along.
Even in these though, their core concepts would remain consistent, and much like the aforementioned formal writeups that occasionally would show up,
these would appear most in large cultural centers.

\subparagraph{The most obvious of the large cultural centers would be any religious nexus.}
Religions are often a driving force behind these manners of cultural artefacts.
What other organizations were large enough at the time to be able to fund productions on the scale of the Sistine Chapel or the various cathedrals that much of Europe is famous for \citep{wiki:christian_art}?
This was especially true after the Black Plague hit Europe.
As families and groups became smaller, wealth consolidated in fewer hands, allowing those particular entities to make commissions and purchases that would have been impractical to any before them \citep{armesto2010}.
Again, games easily being described as a form of art, religions often penned and recorded the rules for various games (or did it for people who had the means to afford it!).
Senat, for instance, was even directly tied into the beliefs of the time --- the naming of several of the tiles were related to what people considered the cycle of life. 
But even relatively secular (or at least, not directly religiously motivated) games would get similar treatments.
Wei Qi was a rough game, at first, that was adopted and held as an example of types of strategies that would get people far in both literal warfare and in the various figurative battlefields in life
(modern sets occasionally even come with excerpts from \emph{The Art of War}, to boot!).
Even the Tarot was used, much later on, as a means of representing the stages of a person's life --- from the Fool to Judgement, one was said to be able to see all of what a person was to go through in the cards. \citep{botermans2008}

\paragraph{Games spread as communication with surrounding regions increased.}
People naturally talk --- pub games tended to spread to various lands when travelers would hit up an inn for rest, and play games with each other to pass the time.
In addition, as people became more mobile, cultural diffusion occured --- concepts that were common in a central location within that culture would tend to spread outwards.
Any type of increase in communication could cause this: trade between various regions, conquest, missionary events.
One would be rather hard pressed to find a time when people could talk with each other and not, somehow, go through some degree of change from it.

\paragraph{The most well-known empire of the time in much of Western culture was the Roman Empire, and this would heavily influence the type of culture that surrounding regions would experience.}
The Roman Empire is traditionally considered the nexus of Western civilisation of that particular time period, and it had a significant impact on the distribution of games.
There was significantly less on the development front from the empire itself --- but this was due to its rather unique nature during that time.
The Roman Empire was particularly famous for being able to take the best out of the various cultures surrounding it into itself, and with regards to games, there was no exception.
Many of the core concepts of games attributed to the Romans showed up in the smaller countries comprising it first, or in its trade partners.
There were few completely original ideas to them \citep{armesto2010}.

\subparagraph{Tabula, the Roman precursor to the modern game now known as backgammon, is perhaps one of the better examples of this effect.}
The game was tied to the yearly cycle and the calendar: twenty-four dots for the hours of a day,
a dozen dots for each month of the year, and thirty chips representing the days of the month \citep{botermans2008}.
The game itself is relatively simple, and it is not clear as to whether these parallels are completely intentional,
but even here, it is evident that the game is heavily influenced by the culture it was created in.
It certainly was not invented here --- it is almost certainly a descendant of Senat (nearly five thousand years old),
and there are many examples of convergent evolution that resulted in similar games (such as an Asian equivalent being attributed to Indian invention),
but again, the important point is that the Romans had a tendency to collect interesting cultural traits from those around them.
This centralization would make spreading such games much easier.

\subparagraph{Games, along with other cultural traits, spread along with the expansion of the Roman Empire.}
This could certainly be said to be associated with the spread of \emph{any} large group of people, but this particular case is very notable because of the aforementioned tendency of the Romans to pick up
whatever looked interesting from those around them.
This meant that any game that was subsumed into the culture at the time would be spread much farther than it normally would as the empire expanded, so would the influence of ideas and memetics within it.
Any culture would have an influence on both those within it, and those next to it;
and the prevalence of Roman culture at the time meant that even China and such countries had some influence on it and were influenced in turn by it \citep{armesto2010}.
The tendency of the Roman culture to expand was only reinforced by its constant stream of conflict and conquest (up until its fall, which was mainly heralded by political events),
and even without the conquests, its extensive trade routes with countries extending out to East Asia and down to Africa ensured that anything the culture held would be known, eventually, to the farthest reaches
of the recorded world at the time.

\paragraph{With this in mind, it becomes very helpful for ethnologists to use games to track how far the influence various cultures held in the world extended.}
After all, if games would be passed on when various people communicated, whether it be some soldiers getting together, some travelers having a drink together, or some farmers being bored and sitting around chatting;
it would be relatively safe to assert that one could check on how fast ideas spread by seeing how far specific games with specific mechanics extended from whatever region they were presumably invented in.
The expansion of the ideas behind the games would not even need to be deliberate --- being around people playing a game is both incentive and a means to pick up the game for oneself.
Again, "diffusion" is an excellent term for this effect, as practices like these games would slowly become more and more commonplace as time went on.

\subparagraph{Going back to the backgammon or tabula example, it could be said to have shown up in almost all cultures that the Egyptions touched, and spread from there.}
It exists in the same family of games as Senat, which is one of the oldest recorded games of its class, having clear roots as far as five thousand years back, and with various related artifacts being found
in tombs or ruins going back to the fourteenth century B.C.E..
There are records of a similar game showing up in China in around 1000 C.E., and we know that there was, in fact, trade with the Roman empire long prior to that.
Those particular records even assert that their backgammon-like game showed up roughly around the 220 C.E. mark, which would make sense if it arrived via cultural diffusion.
From China, that particular game spread out to Japan and Southeast Asia, both showing up with their own variants of it shortly afterwards \citep{botermans2008}.
Thus, we can conclude that there was communication, indirectly, extending from Egypt to Southeast or East Asia.
As it is unthinkable that only games would make it that far, any ethnologist would be able infer that there may have been other information passed along those routes as well,
and such thinking could serve as a useful springboard and start off research avenues to find out more about such cultural transfers.

\subparagraph{In addition, as communication between cultures became easier, this diffusion became more prominent, allowing the aforementioned ethnologists more interesting insights into the various cultures involved.}
Foreign games, to some degree, may even have been popular in various eras simply because they were different.
One does not need to look farther from many modern video games to see this effect (though this is, of course, heavily supported by the ease of communication modern societies have).
Settings ranging from China, to Japan, to the Middle East or common; actually staying within the United States (home, in context of this particular essay) is becoming relatively scarcer! \citep{johnson2009}
The reasoning behind this has changed from then to modern times, however.
It is very likely that a fascination with foreign games would have stemmed from the relative uniqueness of games at the time, or perhaps because the actual physical construction of said games may have been different and interesting,
but the modern usage of foreign tropes or styles in games may be more like to stem from an attempt to be more all-inclusive, or to give the designers and easy means to shorthand certain tropes unique (or percieved to be unique)
to specific cultures or groups.
Either way, the constant increase in the "worldliness" of games is directly tied to how easy communication has been through various eras, giving prospective historians another angle to track changes and influence from.

\paragraph{Thus, games, due to the very nature of their design and dissemination, are excellent ways to examine the influences of the cultures that they passed through.}
As they arose from the thinking and traits of the culture that invented them, and were adapted and modified by all those that played them, they can be said
to reflect both what the thinking of those involved were, and how far the aforementioned thoughts were able to extend outwards to.
It is, in a way, one of the most reliable way of tracking how fast information could spread between cultures --- there were always games, no matter how good or bad any individual group's situation was throughout history.

