\documentclass{article}

\usepackage{times}

\title{HIS150 Test 2}

\author{Marcus Meng}

\begin{document}

\maketitle

\section{Short Answer}

\subsection{Major Ideas and Findings of Two Intellectuals}

Spinoza and Descartes are, perhaps, the two most well-known proponents of the materialist and dualist philosophies of thought in history so far.

Dualism is easily understood as the concept that there is some sort of nonmaterial phenomenom that can influence the world, usually represented in religions and modern thinking as the concept of a "mind" or "spirit" that is somehow seperate from the body.
Descartes believed this to be true, and a consequence of this type of thinking was that the concept of "free will" became popular amongst those that tended to subscribe to this idea.
The problem is, as Spinoza mentioed in his writings, is that there is no rational or measureable way that this could possibly be true.
The gist of the argument is that in general, for something to influence the material world in a nonrandom (nonrandom being an important note, as "free will" cannot exist if all actions are truly random), it must take in information from the material world.
If there is some sort of spirit entity or whatnot that can recieved information from the material world, it by definition is being affected by the world, and thus, is a part of it.
Thus, Spinoza contended, dualistic worldviews are inherently contradictory.

Spinoza and his contemporaries, however, who subscribed to the materialistic worldview, were quite extreme in their definition of how the world worked, though.
Perhaps as a function of the slightly more limited scientific knowledge at the time, they asserted that any measurable pattern must have a literal material presentation -- something that is actually quite difficult to verify.
While this thought worked quite well for certain concepts, such as a physical representation of electrical forces (as we now know about electrons), it is more difficult when the concept is actually a conjunction of several effects, such as "thought" (which, perhaps, may be modelled as a specific series of patterns of electricity in the brain, but is not the brain itself).

Nowadays, the dualistic worldview tends to be the default concept that many western cultures subscribe to, even though it has lost significant popularity amongst modern philosophers. The strict materialistic worldview that Spinoza insisted on is not used, however, the concept that all things exist within a single reality is the basis for many types of scientific thought.

\subsection{Colonies in Military and Politics}

Colonies during this time period were, as such things generally have been, set up as forward outposts to facilitate trade between two areas.

In general, colonies became the springboards by which a culture would then politically or militarily influence another area, sometimes one as a consequence of another.
It would happen even when the target culture would attempt to sequester the colonies, as what happened in China -- the Chinese attempted to keep the American, French, and British trading outposts sealed in small enclaves near the ports, but the Opium Wars and events leading up to them forced the enclaves to open up despite the wishes of the government at the time.

Other notable examples of colonies becoming a major influence on both the target location and the home nation are, of course, the effective colonization of Indian ports by various British companies, and the travelers that eventually settled in America.

\subsection{Roles of Revolutions}

The Age of Enlightenment is often directed to as one of the periods with the most dramatic revolutions, both in modes of thinking and in military terms.

The most obvious was the sudden explosion of philosophical schools of the time.
This could, perhaps, be attributed to far more extensive travel than in previous ages, allowing significant cross-pollination of ideas (such as various countries setting up trading outposts and colonies in other countries), and relatively increased literacy.
Academies in various countries being set up promoted the spread of information, and the relatively lower cost of mass-printing and the subsequent development of an actual print industry made information far more accessible than it had been in the past.
While it is generally thought that the most popular stories then (as they are now) were fairly sensationalist pieces that did not offer learning as much as entertainment, the slowly rising literacy rate (approaching and occasionally breaking 50\% in some areas) would allow societies to grow in ways that would have been difficult before.

Militarily, several revolutions happened that would change how entire countries were run or seen.
The American Revolution, of course, lead up to the USA being an entirely new country for others to deal with; other ones such as the fallout of the Opium Wars would reform countries like China and set them on the path to where they are in modern times (albeit at a great cost in this particular example).

\section{Essay}

The most dominant area of growth during this particular period was in economic terms.
This, in turn, tended to drive the military and political developments, and allowed for intellectual development in various nations.
The various events would then feed back and cause even greater economic growth.

The easiest way to see the influence of various economic factors on political issues was in the trade with China.
Originally, the trade was almost detrimental to Britain.
The merchants being forced to stay in enclaves, and being only allowed to use silver currency instead of actual goods crippled Britain's bargaining position.
Britain, in turn, had great interest in the various things that China would export, and it was this desire to expand and obtain those items that eventually would drive them to dump excess opium from their Indian trading routes into China and spark off the Opium Wars.
Britain then utilized their military might to influence their politicking, and wrangled out substantially better trading agreements with China, thus improving their own economy significantly.

Economic pressures also avoided significant fighting, in some cases.
In many skirmishes with various entities on the American continent, several battles were foregone as the British attempted to hold on to their good islands in the Southeastern region -- they were unwilling to divert ships and manpower to cover what they saw as relatively risky and low-return areas compared to the prime trading advantage the islands would grant.

Moving away from military matters, the significant economic boosts through trade also meant that further specialization of various societies could happen.
By making it so that a given society would only need to produce, say, food that their climates and lands were most appropriate for, and then trading for what they needed, the overall population sizes spiked upwards.
At this point, the supply was generally good enough to keep up with the population increase.
One can point to the well-known Hierarchy of Human Needs as presented by Abraham Maslow.
The relative safety and stability brought about by these various changes would allow academics to flourish, and let many people focus on various intellectual pursuits rather than worry about whether their next meal existed.

Famous scientists such as d'Holbach, Newton, and Lavoisier would take advantage of this atmosphere to promote their works, setting up many of the concepts and methods we use today in modern science. 
Calculus, cartesian geometry, and other such concepts would have been hard pressed to become as well known as they are now if it weren't for the infrastructure in place to promote such ideas.
Thinkers such as Hume, Hobbes, Locke, and Paine would codify and record both the ideas of previous ages and inspire new concepts -- many, such as the concept of a social contract, and the the blueprints for a constitutional republic, are in use almost exactly as presented then.
The face of many modern countries would be extremely different if these ideas had not been able to get out.

Thus, it can be said that economic growth during this period was the dominent factor in the various changes that happened. If it were not for the increased trade, specialization, and industrialization during this time period, the infrastructure and concepts necessary for the political, military, and intellectual pursuits would not have existed at all.

\end{document}

