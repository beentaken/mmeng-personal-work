\section{Introduction}

Newton's second law states that the acceleration that an object undergoes when a force is applied to it is proportional to both the size of the force applied and the mass of the object.

In this experiment, we will attempt to demonstrate that this is a reasonable approximation of kinematics within the systems that we are working in.

\section{Procedure}

The track is set up at an angle to the desk, and a pulley is attached to the high end.
A string is threaded through the pulley, and one end is attached to a collision cart.
The other end hangs downwards and holds a set of weights.

An initial set of weights was chosen arbitrarily, and an equation was derived analytically to estimate what angle the ramp would require to have the cart accelerate at a usably small rate.
The angle was set to that initial value by means of a pair of screws on the ramp.
The cart was let loose and allowed to accelerate up the ramp, and the acceleration recorded with a motion sensor.
The angle was adjusted and recorded until the cart accelerated at the desired rate.

\section{Results and Discussion}

\subsection{Questions}

\begin{itemize}

\item \textbf{Is the acceleration of the motion cart constant? Why or why not?}

For the most part, the acceleration values that our sensors picked up were fairly consistent while the cart was moving.
The small variations in it can reasonably be attributed to inaccuracies in the readings or small variations in the test setup.

\item \textbf{Is it possible to get the cart to ride up the ramp when the hanging mass is less than the mass of the cart? How?}

Definitely --- it happened in the experiment itself.
To do so, the angle must be sufficiently small so that as little of the cart's own mass is opposing the force exerted on it by the tension on the connecting string.

\item \textbf{How did you measure the angle of the track?}

The bottom of the track was measured relative to the angle of the ground, which we assumed to be horizontal, using a protractor.

\item \textbf{What is the acceleration of the cart as a function of cart mass, \begin{math}m_c\end{math}, and hanger mass, \begin{math}m_h\end{math}?}

Assuming that \ensuremath{\theta} is the angle that the ramp is at relative to the ground (which we will assume is horizontal), and that the pulley
and ramp are frictionless, the acceleration of the cart can be modelled by the equation:

\begin{displaymath}a = \frac{m_c g \sin \theta - m_h g}{m_c + m_h}\end{displaymath}

This was, in fact, the basis for the equation used to get the initial angle to begin the experiment from.

\item \textbf{Does your data verify Newton's 2\nd law? How or how not?}

\item \textbf{How would this change if the cart was sliding instead of rolling on wheels?}

If the cart was sliding, the influence of friction on the system would not be quite as small, and would definitely have to be taken into account.
In addition, there are different coefficients for static and kinetic friction, meaning that there will be situations where there would be enough force from the weights to keep the cart moving --- but only if the cart was moving already.
The test mechanism would become rather more complex.

\end{itemize}

\subsection{Error Analysis}

The largest source of error is very, very likely to be the angle measurement itself.
Our measuring mechanism wasn't very stable, and it is very likely that our measurements are only accurate within maybe one degree of the actual value.

With that said, however, our results only required relatively minor adjustments relative to our predicted values to achieve, so the contribution of the error in the measurements may be relatively small.

\section{Conclusion}

As the results obtained from this experiment were consistently within a range, and always off by a similarly small amount from our predicted values, it is likely that the equation used to predict
the acceleration is not inaccurate.
The differences in the actual results and the expected values could probably stem from one of the many minor factors that were assumed to be negligible.

