\documentclass{article}

\usepackage{times}

\title{Characters in L.A. Confidential}

\author{Marcus Meng (FLM151)}

\begin{document}

\maketitle

\begin{abstract}

LA Confidential 1997 Hanson An adaptation with multiple significant characters.

Are there three protagonists? Yes/No? Support your answer. If you answer no, what functions do the other characters play? If you answer yes, explain how they interact without confusing the audience.

\end{abstract}

\emph{L.A. Confidential} is a 1997 movie that, at first glance, appears to star three protagonists.
The term "protagonist" refers to a character whose objectives and emotions the audience feels a sense of empathy with.
Ed Exley, a cunning and idealistic rookie; Bud White, a bully-like detective; and Jack Vincennes, a popular "Hollywood cop"; all find themselves trying to figure out the actual events behind a murder at a local diner.
They each play critical roles in the story, and it is possible to argue that all three of them are actual protagonists in this story.

Exley starts out as what appears to be an idealistic and talented young cop, that uses his excellent success to get a position as a detective -- a position that he's been trying for, in his attempts to follow in his father's footsteps.
He turns out to have significant amounts of cunning, both able to manipulate people to secure promotions and other accolades for himself, and to get into the heads of the various criminals he deals with.
Over the course of the movie, the viewer realizes that his constant push for more and more power is a deviation from his original goal, which was to make sure that what happened to his father -- a murder where the perpetrator got away -- is not repeated.
Exley himself realizes that he screwed up when Vincennes questions him as to his motivations, and the memory of his original goals eventually lets him gather himself and pull the trigger at the end of the movie, when the real criminal is about to get away.

White, on the other hand, is seen by the rest of the squad as a musclebound thug.
He is often called in to intimidate suspects or provide some physical force when finesse doesn't work.
At the beginning of the story, it wouldn't be difficult to assert that he's almost set up as the opposite of Exley -- a succesful detective with the exact opposite of Exley's methods.
However, through the story, the viewers get insights into his motivations, thoughts, and ideals.
He tends to be short tempered, but he also fights for what he believes is right.
When he's manipulated into doing things by Dudley, he becomes angry -- he refrains from killing or maiming Exley despite the planted evidence because he realizes that it's not fully Exley's fault in that situation.
And he's hardly as thick as he appears at first glance -- he figures out that something's wrong with the murder at the restaraunt well before Exley even gets to it.

Vincennes is a showman. He makes a lot of kickbacks off the studio he works for, and he often goes out of his way to be fairly flashy.
As a result, he's well liked by a lot of the people that know him, and his connections let him see how politics in the squad and in the associated organizations often works.
He appears to be motivated by greed at first, but the viewers eventually notice that there's a deeper side to him when Exley confronts him.
He talks about how he's been working on a murder that nobody else seems to even care about; and the implication of the statement is that his showmanship is a cover for a person too tired and jaded by the inefficiency (or, perhaps in this case, outright obstructionary) of the heavily political police organization. He seems to admire Exley's ability to stick to his ideals, and that inspires him to assist Exley's investigation.

By the definition of the protagonist referred to earlier, all of these characters can easily be considered protagonists of this story -- it is easy for a viewer to understand and follow the motivations of all three of these characters.

\begin{thebibliography}{9}

\bibitem{wiki:protagonist}
	Wikipedia: Protagonist, 2010.

\end{thebibliography}

\end{document}

