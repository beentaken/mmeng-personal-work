\documentclass{article}

\usepackage{times}

\title{Structure of XXY}

\author{Marcus Meng (FLM151)}

\begin{document}

\maketitle

\emph{XXY} is a 2007 movie about a young, intersexed teenager named Alex coming to grips with her situation and the perceptions of various people she meets.

Most three-act structures have a few major plot points: in the first act, the protagonist will realize an obstacle or problem stands in the way of his goal, and resolve to take it on;
in the second, the protagonist usually seems close to his goal before additional setbacks are revealed;
and in the third, the protagonist will finally confront the issue at hand, possibly resolve it, and finally the story will wind down into a conclusion.

In \emph{XXY}, the story is very much focused on the characters' relationships with each other, rather than immediate actions, save a few exceptions.
The premise of the story is that Alex's family has moved to that small village to escape persecution on their previous home, due to Alex's unique condition.
The problem of how other people view Alex's state becomes apparant when even they people they called in to help -- the plastic surgeon -- seems to regard Alex as an abnormality that should be fixed.
This sets up the conflict: Alex must decide on what she is willing to live with, to undergo surgery, or to accept and deal with the criticism and curiousity of others around her.
It may be tempting to say that Alex's relationship with Alvaro, or even Alex's immediate conflict with the other people living in the village, is the focus of the story, but no.

Admittedly, the focus on Alex's decision only becomes clearer halfway through the movie, and near the end.
What might be considered the catalytic event is unusually subdued in this movie -- it's likely her meeting with Alvaro that sets off the main plot.
She takes a liking to him, and even asks him early on whether he wants to have sex with her;
this question describes the issue.
He's going to find out what she's like and she'll have to decide what she wants to be, otherwise it'll turn into a situation like what their family is getting away from -- other people attempting to define her for her.

This issue is only resolved near the very end of the movie (the climax happens, and the movie has a very short denouement, as such).
After being attacked on the beach, Alex dumps the rest of her pills on to the floor and tells her mother that she doesn't care about hiding who she is in favor of informing the police about the attack.

Since the focus of the movie is already relatively subtle, the movie is deliberately built with a simple action line so that it does not distract from the character's feelings and emotions.
The actual happenings can basically be summed up as: Alex and family arrive, they invite Alvaro and family over, Alex and Alvaro get together and then fall apart, and Alyx is attacked on the beach.
There is little ambiguity about who did what, there is no over-the-top action, and this is all so that the emotive line can take precedence -- it's where all the story actually happens.

\begin{thebibliography}{9}

\bibitem{web:3act}
	\emph{Three-act Structure}, http://www.cod.edu/people/faculty/pruter/film/threeact.htm

\bibitem{web:eff}
	Eye For Film: \emph{XXY} Movie Review, provided on Moodle.

\bibitem{web:imdb}
	XXY (2007), http://www.imdb.com/title/tt0995829/

\end{thebibliography}

\end{document}

