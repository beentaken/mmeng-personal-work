\documentclass{article}

\usepackage{times}

\title{Camera Work in \emph{The Conformist}}

\author{Marcus Meng (FLM151)}

\begin{document}

\maketitle

\emph{The Conformist} is a 1970 film by director Bernardo Bertolucci that focuses heavily on the main character's dedication (or lack thereof) to the fascist cause, especially when it brings him to have to assassinate his former mentor.

The first picture presented is of when Marcello is attempting to return the gun to Manganiello and get himself out of having to assassinate the Professor.
A lot of the tension in this particular scene is between the two characters -- the gun, despite it's (relatively) heavy focus on the first few shots of the scene, serves only to accentuate Marcello's indecision.
Through the entire scene, the gun itself never breaks the line between both character's eyes as they talk, and the Camera has a tendency to stay slightly behind Manganiello, implying Marcello's inferior position and self-doubt.

The fifth picture shows a scene during a flashback to Marcello's childhood.
Lino basically gets Marcello into his room on pretext of showing him a gun and then attempts to molest Marcello, who then uses the gun and shoots Lino.
The camera in the scene starts out in a fairly straightforward manner, following Marcello and Lino's actions down a corridor and into the room itself.
When the tension rises and Marcello and actually fires the gun, though, the camera turns to go for something in between a close and medium shot, showing each bullet planting itself into the walls, and there are no cuts -- the camera appears to be following Marcello's own line of sight, eventually moving downwards to find the unconcious Lino on the floor.
It essentially places the viewer into Marcello's point of view, so that the audience can almost feel what Marcello is when he realizes what he's done to Lino.

\begin{thebibliography}{9}

\bibitem{web:imdb}
	\emph{The Conformist}, http://www.imdb.com/title/tt0065571/

\end{thebibliography}

\end{document}

