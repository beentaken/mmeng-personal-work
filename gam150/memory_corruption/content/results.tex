\section{Results and Discussion}

\subsection{Questions}

\begin{enumerate}

\item How do the properties on the following list affect how fast an object rolls down a ramp?

\begin{itemize}

\item Surface Friction

Surface Friction mainly affects whether an object rolls, slides, or undergoes some combination of the two motions.
If an object slides more, its acceleration is reduced since there is a frictional force opposing its motion.
If an object rolls, however, some of the potential energy is converted into rotational kinetic energy as well, and thus has a lower overall velocity at its center of mass.

\item Shape

The shape of an object affects where its center of mass is and how its moment of inertia is calculated.
The moment of inertia, in particular, has a heavy influence on how fast an object will accelerate while rolling.

\item Mass

The mass directly influences the acceleration of the object when a force (in this case, gravity mostly) is applied to it, as in standard linear kinematics.

\item Density

The density of an object is important for the same way mass is.
In this particular experiment, it is not particularly different from mass because the objects we used were of uniform density, or close enough as to make no odds.

\item Ramp Angle

The ramp angle influences the magnitude of the component of acceleration due to gravity applied to the object.
As the angle increases, a larger acceleration is applied.

\end{itemize}

\item Can you think of other properties which affect rolling?

For a simplistic experiment like this, there are no other notable properties that may heavily influence the experiment.
In theory, there are such things as drag forces and the possible kinetic friction, but drag forces are negligible at this scale and we are assuming that our objects our not sliding, so kinetic friction is discounted.

\item Are there any of the properties that affect the rolling that you are neglecting?
Why are you able to neglect them?

In this particular experiment, friction is not directly accounted for, in sense that it is not measured and factored directly into the calculations.
Instead, we are assuming that the coefficient of static friction is sufficiently high to ensure that all of the objects in question are rolling, and not sliding, down the ramp.

\item Calculate the translational kinetic energy and change in potential energy at a point where the object is partway down the ramp.
Where is the missing energy?

The missing energy is in the rotational kinetic energy --- since the objects are not simply sliding down the ramp, part of the potential energy has gone into the rotational motion of the object in question.

\end{enumerate}

\subsection{Error Analysis}

Relative to most of the experiments conducted this semester, this one is relatively simple.
The largest sources of error may come from the data measurement itself.
We attempted to filter our data by slicing off the ends where the motion sensor's readings were inaccurate, since the sensor has trouble reading data too close and too far from it, but it is unclear as to whether we caught
and good data in the process, as well.

