\documentclass{article}

\usepackage{cite}
\usepackage{times}
\usepackage{url}
%\usepackage{datatool}

\title{PHY200L Lab 04 -- Relative Velocity}

\author{Marcus Meng}

\date{October 7, 2010}

\begin{document}

\maketitle

\section{Introduction}

In this experiment, we attempted to determine the effects of an angled track on the velocity of two moving carts.

\section{Procedure}

Our procedure differed from the provided procedure on the original lab sheet due to the lack of working motion sensors at the time of the experiment.

\subsection{Horizontal Track}

We set up the track and leveled it by seeing whether a cart would roll if left in place on it.
Sheets of paper were stacked under one end of the track to get the structure level.

One end of the track was placed against a wall, so that the wall could serve as a structurally sound starting board for the plunger cart to push off of.
A reading was taken at the front of the cart, with the plunger fully compressed, before the cart was released, to serve as a mark for the starting point of the cart's movement.
A barrier (in this case, a ruler) was placed exactly one meter from the starting point of the cart, to act as a stop position.

In each reading taken, the cart was launched using the plunger at the same time a stopwatch was activated.
The cart would move along the track, and when it collided with the ruler at the one-meter mark on the track, the stopwatch was stopped and the time taken.

\subsection{Angled Track}

For the second part of the experiment, we raised one end of the track, so that the track would be angled down towards the wall.
The elevation of the higher end of the track was recorded.

A second cart was placed near the top of the track, and its starting position marked.
The plunger cart was, once again, placed near the wall.

In each reading taken, the upper cart was released to allow it to roll down the track with no additional acceleration outside of gravity provided.
At the same time, the plunger cart was launched and the stopwatch started.

The point where both carts collided, as well as the time they collided at, was recorded.

\section{Results and Discussion}

%\DTLloaddb{horizontal}{horizontal.csv}

%\begin{table}[htbp]
%	\caption{Horizontal Track Data}
%	\centering
%	\DTLdisplaydb{horizontal}
%\end{table}

\subsection{Questions}

\begin{itemize}

\item What is the average intial velocity of cart A? Include the standard deviation in your result.

The average initial velocity was about 72.2 cm/s, with a standard deviation of 11.2 cm/s.
The initial velocity should be near the average velocity, since there is little friction on the horizontal track tests.

\item What is the velocity of cart A relative to the track?

Roughly 72.2 cm/s.

\item What is the relative velocity of cart A to cart B?

In the first test, cart B is stationary, so cart A is moving at roughly 72.2 cm/s with regards to cart B.

In the angled track tests, however, the relative velocities of both carts should be near the average velocity of both carts summed up.
Thus, the average relative velocities should be roughly 78.0 cm/s.

\item What would you change if you angled the track the other way?

Presumably, the normal (non-plunger) cart would cease to move at all, being at the bottom of the track and pressed against the wall already.
The plunger cart would have an even higher average velocity, seeing as it will be accelerating down the track due to gravity the entire way.

\end{itemize}

\subsection{Errors}

Unlike previous labs, most of the measurements were done by hand, potentially leading to large sources of error stemming from deficiencies in human reaction time or measuring ability.
The measurement most subject to this would be the time, since it was taken by hand based on when we heard the carts striking either the ruler or each other.
Mean human reaction times for audible stimuli are around the 160 millisecond mark, and visual stimuli appear to hit a roughly 190 millisecond turnaround\cite{kosinski2010}.

We did not have an accessible means to compensate for this --- there was no practice or additional utilities involved in the timing, and the delay due to reaction time may have been introduced twice:
once when the timer was started, and once when it was stopped.
As such, given some uncertainty, we can assume that the measurements are no more accurate than within half a second of the actual values.

Additionally, we had no means of making sure the horizontal track was actually level.
Our approximation of it seemed to work reasonably, but a bubble level or similar tool would have been useful.
The cart would occasionally move on its own when placed on different parts of the track --- either the track was not level, or perhaps the track was slightly warped.

\section{Conclusion}

The original experiment was intended to be an analysis of how relative velocity of the two carts affected measurements.
This goal was not directly doable using the equipment at hand, however, an analytical derivation of the results is still obtainable.

Since we measured the cart's velocity relative to the track, and at least, on the horizontal track part of the experiment, the second cart was motionless relative to the track,
we could infer that the first cart's velocity relative to the second cart was the same as its velocity relative to the track.

This is more difficult to set up for the second part of the experiment without basically stating what was known ahead of time with regards to the experiment,
but the relative velocity between the carts turns out to be based on the difference between the individual velocities of the carts,
assuming that they are placed along the same coordinate scale.

% Also, IT sucks.

\bibliographystyle{plain}
\bibliography{lab04}

\end{document}

